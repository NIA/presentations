\documentclass[a4paper, 12pt]{extarticle}
  \usepackage{cmap}
  \usepackage[hidelinks,pdftex,unicode]{hyperref}
  \usepackage{mathtext} % для кириллицы в формулах
  \usepackage[T2A]{fontenc}
  \usepackage[utf8]{inputenc}
  \usepackage[russian]{babel}
  \usepackage{indentfirst}
  \usepackage{cite}
  \usepackage{amsmath} % для \eqref
  \usepackage{amssymb} % для \leqslant
  \usepackage{amsthm} % для \pushQED
  \usepackage{gensymb} % для \degree
  \usepackage[usenames,dvipsnames]{xcolor}
    \definecolor{dkblue}{rgb}{0,0,0.6}
  \usepackage[pdftex]{graphicx}
  \usepackage{subfig}
  \usepackage{tikz}
    \usetikzlibrary{positioning,fit,shapes,calc}
  \usepackage{numprint}
  \usepackage[left=1.5cm,right=1.5cm,top=1cm,bottom=1.5cm,bindingoffset=0cm]{geometry}
  \usepackage{datetime}
  \usepackage[normalem]{ulem} % красивые подчёркивания, но обычный курсив в \em, \emph
    % Хак для использования \sout итп в заголовках (http://tex.stackexchange.com/a/22415)
    \DeclareRobustCommand{\hsout}[1]{\texorpdfstring{\sout{#1}}{#1}}
  \graphicspath{{../img/}{../../img/}}
  \frenchspacing

  \DeclareSymbolFont{T2Aletters}{T2A}{cmr}{m}{it} % кириллица в формулах курсивом

  \addto\captionsrussian{
    \renewcommand\contentsname{Содержание}
    % перекрываю \refname, чтобы список литературы сам добавлял себя в оглавление
    \let\oldrefname\refname
    \renewcommand\refname{\addcontentsline{toc}{section}{\oldrefname}\oldrefname}
  }

  \newcommand{\underscore}[1]{\hbox to#1{\hrulefill}}
  \newcommand{\todo}[1]{\textbf{\textcolor{red}{TODO: #1}}}
  \newcommand{\note}[1]{\textit{Примечание: #1}}
  \newcommand{\eng}[1]{{\English #1}}

  % обёртка с моими настройками поверх figure:
  % \begin{myfigure}{подпись}{fig:label} ... \end{myfigure}
  \newenvironment{myfigure}[2]%
    {\pushQED{\caption{#1} \label{#2}} % push caption & label
     \begin{figure}[!htb]\centering } %
    {  \popQED % pop caption & label
     \end{figure}}

  % вставка картинки: \figure[params]{подпись}{file}
  % создаёт label вида fig:file
  \newcommand{\includefigure}[3][]{
    \begin{myfigure}{#2}{fig:#3}
      \includegraphics[#1]{#3}
    \end{myfigure}
  }

  % вставка subfigure внутри myfigure:
  % \subfigure[params]{подпись}{file}
  \newcommand{\subfigure}[3][]{
    \subfloat[#2]{\label{fig:#3}\includegraphics[#1]{#3}}
  }

  \newcommand{\vect}[1]{\vec{#1}} % единое выделение векторов (стрелкой)
  \newcommand{\matx}[1]{\mathbf{#1}} % единое выделение матриц (полужирным)
  \newcommand{\transposed}{\top} % единый знак транспонирования (U+22A4 down tack)
  \newcommand{\conj}[1]{#1^*} % единое обозначение комплексного сопряжения (черта сверху)
  \renewcommand{\le}{\leqslant} % <= с наклонной нижней перекладиной
  \renewcommand{\ge}{\geqslant} % >= с наклонной нижней перекладиной
  \renewcommand{\phi}{\varphi} % phi завитушкой
  \newcommand{\I}{\mathrm{I}} % Интенсивность

  \linespread{1.2}

  % русские буквы для списков и частей рисунка
  \renewcommand{\theenumii}{(\asbuk{enumii})}
  \renewcommand{\labelenumii}{\asbuk{enumii})}
  \renewcommand{\thesubfigure}{\asbuk{subfigure}}

  \setcounter{tocdepth}{3} % глубина оглавления

  \bibliographystyle{gost780u}

  \hyphenation{англ} % убрать перенос в этом сокращении

  % алиас и настройки для numprint
  \newcommand{\num}[1]{\numprint{#1}}
  \npthousandsep{\,}
  \npthousandthpartsep{}
  \npdecimalsign{,}

  \newcommand{\checkdate}[3]{({\Russian дата обращения: \formatdate{#1}{#2}{#3}})}

  % выделение ещё более важных моментов, чем \emph
  \newcommand{\strong}[1]{\textbf{#1}}
  % подчёркивание определяемого в определении
  \newcommand{\define}[1]{\uwave{#1}}

  % участок с увеличенными полями
  \newenvironment{indented}%
    { \begingroup %
        \noindent %
        \leftskip2em %
        \rightskip\leftskip }%
    { \par\endgroup }

  % выделение примеров
  \newenvironment{example}%
    { \begin{indented} %
        \color{dkgreen} %
        \small %
        \textbf{\underline{Пример:}} }%
    { \end{indented} }

  % выделение дополнительной, "продвинутой" информации, необязательной при первом прочтении
  \newenvironment{extrainfo}%
    { \begin{indented} %
        \color{dkblue} %
        \small %
        \textbf{\textcircled{\footnotesize i}} }%
    { \end{indented} }

  % пол-страничная по ширине область (для листингов)
  \newenvironment{halfpage}%
    {\noindent\begin{minipage}[h]{0.49\linewidth}} %
    {\end{minipage}\hfill}

  \author{Иван\,Новиков, напр. физика, пр. инф.\,процессы и системы}
  \title{Интерференция}

\begin{document}

  \maketitle

  \vspace{1.3cm}

  \tableofcontents

  \section{Понятие интерференции}
  \define{Интерференция волн}~--- взаимное увеличение или уменьшение результирующей амплитуды двух
  или нескольких когерентных волн, одновременно распространяющихся в пространстве. Сопровождается
  чередованием максимумов (пучностей) и минимумов (узлов) интенсивности в пространстве. Результат
  интерференции (интерференционная картина) зависит от разности фаз накладывающихся волн.

  Обычно под интерференционным эффектом понимается отличие результирующей интенсивности
  волнового поля от~суммы интенсивностей исходных волн:
  $$ \I_{1+2} \neq \I_1 + \I_2 $$

  При интерференции энергия волн перераспределяется в пространстве. Это не противоречит закону
  сохранения энергии потому, что в среднем, для большой области пространства, энергия результирующей
  волны равна сумме энергий интерферирующих волн. При наложении некогерентных волн средняя
  величина квадрата амплитуды (то есть интенсивность результирующей волны) равна сумме квадратов
  амплитуд (интенсивностей) накладывающихся волн. Энергия результирующих колебаний каждой точки
  среды равна сумме энергий её колебаний, обусловленных всеми некогерентными волнами в отдельности.
  Именно отличие результирующей интенсивности волнового процесса от суммы интенсивностей его
  составляющих и есть признак интерференции

  Интерференция- одно из основных свойств волн любой природы (упругих, электромагнитных, в т. ч.
  световых, поверхностных и др.), и такие характерные волновые явления, как излучение,
  распространение и дифракция, тоже связаны с интерференцией. Нагляднее всего интерференция
  проявляется на \emph{световых} волнах, которые и будут рассмотрены далее.

  \subsection{Механизм интерференции}

  Механизм интерференции проще всего понять на двух крайних случаях сложения волн: интерференционном
  усилении и интерференционном гашении.
  \begin{myfigure}{Интерференционное усиление}{fig:mech-plus}
    \begin{tikzpicture}
      \draw[red, thick, smooth, domain=0:4.5, samples=50] plot (\x, {0.3*sin(8*\x r)});
      \node (plus) at (2, 0.5) {+};
      \draw[red, thick, smooth, domain=0:4.5, samples=50] plot (\x, {1 + 0.3*sin(8*\x r)});
      \node (equals) at (5, 0.5) {=};
      \draw[red, ultra thick, smooth, domain=5.5:9.8, samples=50] plot (\x, {0.5 + 0.6*sin(8*\x r)});
    \end{tikzpicture}
  \end{myfigure}
  Интерференционное усиление происходит, когда складываются две волны одинаковой частоты, пришедшие
  в фазе. При этом амплитуда становится равна сумме амплитуд этих волн, а интенсивность~--- квадрату
  этой суммы (рис.~\ref{fig:mech-plus}).
  \begin{myfigure}{Интерференционное гашение}{fig:mech-minus}
    \begin{tikzpicture}
      \draw[red, thick, smooth, domain=0.4:4.9, samples=50] plot (\x, {0.3*sin( (8*\x + pi) r)});
      \node (plus) at (2, 0.5) {+};
      \draw[red, thick, smooth, domain=0:4.5, samples=50] plot (\x, {1 + 0.3*sin(8*\x r)});
      \node (equals) at (5, 0.5) {=};
      \draw[red, ultra thick, smooth, domain=5.5:9.8, samples=2] plot (\x, {0.5 + 0*\x}); % !!  samples = 2 - оптимизация для прямой!
    \end{tikzpicture}
   \end{myfigure}
  Напротив, интерференционное гашение имеет место, когда одна волна отстаёт от другой на половину
  периода, то есть на $\pi$ по фазе. Если их амплитуды равны, то получается, что вторая волна
  в точности равна первой, умноженной на $-1$, поэтому суммарная интенсивность равна нулю (рис.~\ref{fig:mech-minus}).

  \subsection{Математическое обоснование}

  \begin{itemize}
    \item Свет~--- электромагнитная волна
    \item Рассмотрим плоскую электромагнитную волну, в ней напряжённость электрического поля
      \[
        \vect{E}(t, \vect{r}) = \vect{E}_0 \cdot e^{i(\omega t + (\vect{k}\cdot\vect{r}) + \phi)}
      \]
    \item Интенсивность определяется квадратом модуля напряжённости поля: $\I = |\vect{E}|^2 = E\cdot\conj{E}$
    \item При сложении двух таких волн поле: $\vect{E} = \vect{E}_1 + \vect{E}_2$ (принцип суперпозиции)
    \item Тогда интенсивность получаемой волны:
      \[
        \I = |\vect{E}_1 + \vect{E}_2|^2 = E_1^2 + E_2^2 + (\vect{E}_1\conj{\vect{E}_2} + \conj{\vect{E}_1}\vect{E}_2) =
      \]\[
        = \I_1 + \I_2 + 2 \vect{E}_{01} \vect{E}_{02} \cdot \cos \left( (\vect{k}_1\cdot\vect{r}) -
          (\vect{k}_2\cdot\vect{r}) + \phi_1 - \phi_2 \right)
      \]
    \item В одномерном случае $\vect{r} = (x, 0, 0)$ и при сонаправленных поляризациях:
      \[
        \I = \I_1 + \I_2 + 2 \sqrt{I_1 I_2} \cdot \cos \left(
          (k_{1_x} - k_{2_x}) \cdot x
        + \phi_1 - \phi_2 \right)
      \]
    \item Возникают чередующиеся полосы с шагом $h=\frac{2\pi}{k_{1_x}-k_{2_x}}$:
      \begin{center}
        \begin{tikzpicture}
          \foreach \x in {1, ..., 20}
          \shade[left color=white, right color=white, middle color=red!75] (0.2*\x, 0) rectangle (0.2 + 0.2*\x, 2);
        \end{tikzpicture}
      \end{center}
  \end{itemize}

  Отдельного рассмотрения требует случай различных частот, тогда в сумме остаётся зависимость от
  времени, и в этом случае под интенсивностью понимается усреднённый по времени квадрат амплитуды:
    \[
      \I = {<{E}}^2{>}_\tau
    \]
  Тогда если волны имеют частоты $\omega_1 \neq \omega_2$, то есть напряжённость поля задаётся
  выражениями:
    \[
      \vect{E}_1=\vect{E}_{01}\cdot e^{i({\omega}_{1}t + (\vect{k}_1 \cdot \vect{r}_1) + {\phi}_1)},
      \hspace{5mm}
      \vect{E}_2=\vect{E}_{02}\cdot e^{i({\omega}_{2}t + (\vect{k}_2 \cdot \vect{r}_2) + {\phi}_2)}
    \]
  Квадрат суммы этих амплитуд равен:
  \[
    E^2=E^2_{1_0}+E^2_{2_0}+2\vect{E}_{1_{0}}\vect{E}_{2_{0}}
    \cdot
    \cos(\Delta\omega t+\Delta{\vect{k}\vect{r}}+\Delta\varphi)
  \]
  С учётом определения интенсивности можно перейти к следующему выражению:
  \[
    \I=\frac{1}{\tau} \int_{t_0}^{t_0+\tau} E^2 dt = I_1+I_2+2\frac{\vect{E}_{1_{0}}\vect{E}_{2_{0}}}{\tau} \int_{t_0}^{t_0+\tau} \cos(\Delta\omega t+\Delta{\mathbf {kr}}+\Delta\varphi) dt
  \]
  Взятие интеграла по времени и применение формулы разности синусов даёт следующее выражения для распределения интенсивности
  \[
    \I = \I_1 + \I_2 + 2(\vect{E}_{01} \cdot \vect{E}_{02}) \cos \left(
      \Delta\omega (t_0+\frac{\tau}{2}) + \Delta \vect{k}\vect{r} + \Delta\phi
    \right) \operatorname{sinc}(\frac{\Delta\omega\tau}{2})
  \]
  С учётом этой записи, можно сформулировать требования для наличия интерференционного эффекта

  \paragraph{Условия возникновения интерференции}
  \begin{enumerate}
    \item Поляризации \emph{не} должны быть перпендикулярны\\
      {(иначе $(\vect{E}_{01} \cdot \vect{E}_{02}) = 0$)}
      \vspace{2mm}
    \item Частоты не должны существенно отличаться\\
      {(иначе $\operatorname{sinc}(\frac{\Delta\omega\tau}{2}) \approx 0$)}
      \vspace{2mm}
    \item Волны должны быть \emph{когерентны}\\
      {(иначе $\Delta\phi$ меняется случайным образом и теряется при усреднении)}
  \end{enumerate}
  
  Перейдём к рассмотрению различных способов наблюдения интерференции света.

  \section{Наблюдение интерференции}
  \subsection{Опыт Юнга}
  В опыте пучок света направляется на непрозрачный экран-ширму с двумя параллельными прорезями,
  позади которого устанавливается проекционный экран. Этот опыт демонстрирует интерференцию света,
  что является доказательством волновой теории. Особенность прорезей в том, что их ширина
  приблизительно равна длине волны излучаемого света. Ниже рассматривается влияние ширины прорезей
  на интерференцию.  Если исходить из того, что свет состоит из частиц (корпускулярная теория
  света), то на проекционном экране можно было бы увидеть только две параллельных полосы света,
  прошедших через прорези ширмы. Между ними проекционный экран оставался бы практически
  неосвещенным.  С другой стороны, если предположить, что свет представляет собой распространяющиеся
  волны (волновая теория света), то, согласно принципу Гюйгенса, каждая прорезь является источником
  вторичных волн.  Если вторичные волны достигнут линии в середине проекционного экрана, находящейся
  на равном удалении от прорезей, синхронно и в одной фазе, то на серединной линии экрана их
  амплитуды прибавятся, что создаст максимум яркости. То есть, максимум яркости окажется там, где
  согласно корпускулярной теории, яркость должна быть практически нулевой. Корпускулярная теория
  света является неверной, когда прорези достаточно тонкие, создавая тем самым интерференцию.  На
  определенном удалении от центральной линии, напротив, волны окажутся в противофазе — их амплитуды
  компенсируются, что создаст минимум яркости (темная полоса). По мере дальнейшего удаления от
  средней линии яркость периодически изменяется, возрастая до максимума и снова убывая.  На
  проекционном экране получается целый ряд чередующихся интерференционных полос, что и было
  продемонстрировано Томасом Юнгом.

  \subsection{Кольца Ньютона}
  Кольца Ньютона — кольцеобразные интерференционные максимумы и минимумы, появляющиеся вокруг точки
  касания слегка изогнутой выпуклой линзы и плоскопараллельной пластины при прохождении света сквозь
  линзу и пластину.  Интерференционная картина в виде концентрических колец (колец Ньютона)
  возникает между поверхностями одна из которых плоская, а другая имеет большой радиус кривизны
  (например, стеклянная пластинка и плосковыпуклая линза). Исаак Ньютон исследовав их в
  монохроматическом и белом свете обнаружил, что радиус колец возрастает с увеличением длины волны
  (от фиолетового к красному).

  \subsection{Интерференция в тонких пленках}
  Получить устойчивую интерференционную картину для света от двух разделённых в пространстве и
  независящих друг от друга источников света не так легко, как для источников волн на воде. Атомы
  испускают свет цугами очень малой продолжительности, и когерентность нарушается. Сравнительно
  просто такую картину можно получить, сделав так, чтобы интерферировали волны одного и того же
  цуга. Так, интерференция возникает при разделении первоначального луча света на два луча при
  его прохождении через тонкую плёнку, например плёнку, наносимую на поверхность линз у
  просветлённых объективов. Луч света, проходя через плёнку толщиной $d$, отразится дважды — от
  внутренней и наружной её поверхностей. Отражённые лучи будут иметь постоянную разность фаз, равную
  удвоенной толщине плёнки, отчего лучи становятся когерентными и будут интерферировать. Полное
  гашение лучей произойдет при $d = \frac{\lambda}{4}$, где $\lambda$ — длина волны. Если $\lambda = 500$ нм,
  то толщина плёнки равняется 550:4=137,5 нм.

  Рассмотрим теперь, как явление интерференции может применяться на практике

  \section{Применения интерференции}

  \subsection{Определение показателя преломления}
  Измерение значения абсолютного показателя преломления веществ основано на свойстве смещения
  интерференционной картины двух когерентных источников волн в зависимости от разности начальных фаз
  их колебаний.

  Рассмотрим устройство измерения значения абсолютного показателя преломления веществ в
  интерферометре Жамена. В интерферометре
  Жамена свет от когерентного источника с длиной волны $\lambda$ с помощью светоделительного устройства,
  представляющего собой линзу, освещающую две щели в непрозрачном экране, разделяется на два
  параллельных пучка. 

  Один световых пучков проходит до щели в экране кювету длиной $l$ с исследуемым веществом, имеющим
  неизвестный показатель преломления $n$, а другой через кювету той же длины, внутри которой воздух.
  Оба световых пучка при облучении каждым соответствующей щели в непрозрачном экране имеют
  оптическую разность хода $\Delta l_{opt}$, вычисляемую по формуле $\Delta l_{opt}=l (n-1)$.
  
  Учитывая, что оптической разности хода лучей соответствует определённое значение разности
  начальных фаз колебаний световых волн интерферирующих световых пучков, следует ожидать смещения
  интерференционной картины на некоторое число интерференционных полос $N_{int}$ относительно направления
  центрального максимума, который имел место для кювет, заполненных воздухом. Поскольку каждой из
  интерференционных полос соответствует оптическая разность хода, равная длине волны $\lambda$, значение
  показателя преломления исследуемого вещества может быть найдено по формуле:
    $$
      n = 1 + N_{int} \frac{\lambda}{l}
    $$

  \subsection{Определение размеров звёзд}
  Задача измерения углового размера источников излучения имеет большое практическое значение для решения многих научных и прикладных проблем. Рассмотрим определение углового размера звезды, представляющей собой естественный источник оптического излучения.

  Для измерения углового размера источника используется свойство пространственной когерентности света, согласно которому
  наблюдение интерференционной картины, создаваемой двумя щелями, расстояние между которыми
  равно $d$ и освещаемые светом длиной волны $\lambda$, возможно, если $d < 2 \rho_c$, где
  $\rho_c$~--- радиус пространственной когерентности источника света. В противном случае интерференционная картина перестаёт наблюдаться.

  Устройства, для проведения измерений в которых используется явление интерференции волн на основе наблюдения их интерференционной картины, называются интерферометрами.

  \subsection{Просветление оптики}
  Одной из главных задач, возникающих при построении различных оптических и антенных устройств СВЧ
  диапазона, является уменьшение потерь интенсивности света, мощности потока электромагнитной
  энергии при отражении от поверхностей линз, обтекателей антенн и пр. приборов, используемых для
  преобразований световых и радиоволн в разнообразных приборах фотоники, оптоэлектроники и
  радиоэлектроники.

  Рассмотрим решение этой задачи на примере "просветления" оптики. Как показывают расчеты, отражение
  света от поверхности линзы сопровождается уменьшением его интенсивности примерно на 4\%. Учитывая,
  что современные оптические устройства, в частности современные оптоэлектронные приборы содержат
  достаточно большое количество линз, зеркал, светоделительных устройств, потери интенсивности
  проходящей световой волны без применения специальных мер могут стать значительными.

  Для уменьшения потерь на отражение используется покрытие оптических деталей пленкой со специальным
  образом подобранными толщиной $d$ и показателем преломления $n$.  Идея уменьшения интенсивности
  отраженного света от поверхности оптических деталей состоит в Интерференционном гашении волны,
  отраженной от внешней поверхности покрытия, волной отражённой от внутренней.  Амплитуды обеих
  должны быть равны, а фазы отличаться на $\pi$.

  \subsection{Голография}

  Голография~--- набор технологий для точной записи, воспроизведения и переформирования волновых полей.

  Голограмма является записью интерференционной картины, поэтому важно, чтобы длины волн (частоты)
  объектного и опорного лучей с максимальной точностью совпадали друг с другом, и разность их фаз не
  менялась в течение всего времени записи (иначе на пластинке не запишется чёткой картины
  интерференции). Поэтому источники света должны испускать электромагнитное излучение с очень
  стабильной длиной волны в достаточном для записи временном диапазоне.

  Крайне удобным источником света является лазер. До изобретения лазеров голография практически не
  развивалась (вместо лазера использовали очень узкие линии в спектре испускания газоразрядных ламп,
  что очень затрудняет эксперимент). На сегодняшний день голография предъявляет одни из самых
  жёстких требований к когерентности лазеров.

  Чаще всего когерентность принято характеризовать длиной когерентности — той разности оптических
  путей двух волн, при которой контраст интерференционной картины уменьшается в два раза по
  сравнению с интерференционной картиной, которую дают волны, прошедшие от источника одинаковое
  расстояние. Для различных лазеров длина когерентности может составлять от долей миллиметра (мощные
  лазеры, предназначенные для сварки, резки и других применений, нетребовательных к этому параметру)
  до сотен и более метров (специальные, так называемые одночастотные лазеры).

\end{document}
