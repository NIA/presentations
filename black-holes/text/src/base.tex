\documentclass[a4paper, 12pt]{extarticle}
  \usepackage{cmap}
  \usepackage[hidelinks,pdftex,unicode]{hyperref}
  \usepackage{mathtext} % для кириллицы в формулах
  \usepackage[T2A]{fontenc}
  \usepackage[utf8]{inputenc}
  \usepackage[russian]{babel}
  \usepackage{indentfirst}
  \usepackage{cite}
  \usepackage{amsmath} % для \eqref
  \usepackage{amssymb} % для \leqslant
  \usepackage{amsthm} % для \pushQED
  \usepackage{gensymb} % для \degree
  \usepackage{color}
    \definecolor{dkblue}{rgb}{0,0,0.6}
  \usepackage[pdftex]{graphicx}
  \usepackage{subfig}
  \usepackage{numprint}
  \usepackage[left=1.5cm,right=1.5cm,top=1cm,bottom=1.5cm,bindingoffset=0cm]{geometry}
  \usepackage{datetime}
  \usepackage[normalem]{ulem} % красивые подчёркивания, но обычный курсив в \em, \emph
    % Хак для использования \sout итп в заголовках (http://tex.stackexchange.com/a/22415)
    \DeclareRobustCommand{\hsout}[1]{\texorpdfstring{\sout{#1}}{#1}}
  \graphicspath{{../img/}{../../img/}}
  \frenchspacing

  \DeclareSymbolFont{T2Aletters}{T2A}{cmr}{m}{it} % кириллица в формулах курсивом

  \addto\captionsrussian{
    \renewcommand\contentsname{Содержание}
    % перекрываю \refname, чтобы список литературы сам добавлял себя в оглавление
    \let\oldrefname\refname
    \renewcommand\refname{\addcontentsline{toc}{section}{\oldrefname}\oldrefname}
  }

  \newcommand{\underscore}[1]{\hbox to#1{\hrulefill}}
  \newcommand{\todo}[1]{\textbf{\textcolor{red}{TODO: #1}}}
  \newcommand{\note}[1]{\textit{Примечание: #1}}
  \newcommand{\eng}[1]{{\English #1}}

  % обёртка с моими настройками поверх figure:
  % \begin{myfigure}{подпись}{fig:label} ... \end{myfigure}
  \newenvironment{myfigure}[2]%
    {\pushQED{\caption{#1} \label{#2}} % push caption & label
     \begin{figure}[!htb]\centering } %
    {  \popQED % pop caption & label
     \end{figure}}

  % вставка картинки: \figure[params]{подпись}{file}
  % создаёт label вида fig:file
  \newcommand{\includefigure}[3][]{
    \begin{myfigure}{#2}{fig:#3}
      \includegraphics[#1]{#3}
    \end{myfigure}
  }

  % вставка subfigure внутри myfigure:
  % \subfigure[params]{подпись}{file}
  \newcommand{\subfigure}[3][]{
    \subfloat[#2]{\label{fig:#3}\includegraphics[#1]{#3}}
  }

  \newcommand{\vect}[1]{\vec{#1}} % единое выделение векторов (стрелкой)
  \newcommand{\matx}[1]{\mathbf{#1}} % единое выделение матриц (полужирным)
  \newcommand{\transposed}{\top} % единый знак транспонирования (U+22A4 down tack)
  \renewcommand{\le}{\leqslant} % <= с наклонной нижней перекладиной
  \renewcommand{\ge}{\geqslant} % >= с наклонной нижней перекладиной

  \linespread{1.2}

  % русские буквы для списков и частей рисунка
  \renewcommand{\theenumii}{(\asbuk{enumii})}
  \renewcommand{\labelenumii}{\asbuk{enumii})}
  \renewcommand{\thesubfigure}{\asbuk{subfigure}}

  \setcounter{tocdepth}{3} % глубина оглавления

  \bibliographystyle{gost780u}

  \hyphenation{англ} % убрать перенос в этом сокращении

  % алиас и настройки для numprint
  \newcommand{\num}[1]{\numprint{#1}}
  \npthousandsep{\,}
  \npthousandthpartsep{}
  \npdecimalsign{,}

  \newcommand{\checkdate}[3]{({\Russian дата обращения: \formatdate{#1}{#2}{#3}})}

  % выделение ещё более важных моментов, чем \emph
  \newcommand{\strong}[1]{\textbf{#1}}
  % подчёркивание определяемого в определении
  \newcommand{\define}[1]{\uwave{#1}}

  % участок с увеличенными полями
  \newenvironment{indented}%
    { \begingroup %
        \noindent %
        \leftskip2em %
        \rightskip\leftskip }%
    { \par\endgroup }

  % выделение примеров
  \newenvironment{example}%
    { \begin{indented} %
        \color{dkgreen} %
        \small %
        \textbf{\underline{Пример:}} }%
    { \end{indented} }

  % выделение дополнительной, "продвинутой" информации, необязательной при первом прочтении
  \newenvironment{extrainfo}%
    { \begin{indented} %
        \color{dkblue} %
        \small %
        \textbf{\textcircled{\footnotesize i}} }%
    { \end{indented} }

  % пол-страничная по ширине область (для листингов)
  \newenvironment{halfpage}%
    {\noindent\begin{minipage}[h]{0.49\linewidth}} %
    {\end{minipage}\hfill}

  \author{Иван\,Новиков, напр. физика, пр. инф.\,процессы и системы}
  \title{Чёрные дыры}

\begin{document}

  \maketitle

  \vspace{1.3cm}

  \tableofcontents

  \section{Введение}

  \subsection{Область применения термина}

  Звёздное небо над головой поражало своей красотой и необъятностью людей ещё с древнейших времён.
  Размышляли над ним и философы (достаточно вспомнить Канта: \emph{<<Две вещи наполняют душу всегда
  новым и все более сильным удивлением и благоговением, чем чаще и продолжительнее мы размышляем о
  них, — это звёздное небо надо мной и моральный закон во мне>>}), и учёные. В итоге с чем с древних
  времён начинает развиваться астрономия.

  Один из самых захватывающих процессов, изучаемых астрономией,~--- это эволюция звёзд. Имея в
  распоряжении огромное количество звёзд в разных стадиях своей эволюции, в ходе наблюдений и
  компьютерного моделирования учёные смогли определить условия, от которых зависит то, по какому
  сценарию будет развиваться звезда, и, самое интересное, чем закончится её жизнь.

  По прошествии определённого времени — от миллиона до десятков миллиардов лет, в зависимости от
  начальной массы — звезда истощает водородные ресурсы ядра. В больших и горячих звёздах это
  происходит гораздо быстрее, чем в маленьких и более холодных. Истощение запаса водорода приводит к
  остановке термоядерных реакций.

  Без давления, возникавшего в ходе этих реакций и уравновешивавшего собственное гравитационное
  притяжение звезды, звезда снова начинает сжатие, как уже было раньше, в процессе её формирования.
  Температура и давление снова повышаются, но, в отличие от стадии протозвезды, до гораздо более
  высокого уровня. Коллапс продолжается до тех пор, пока при температуре приблизительно в 100
  миллионов К не начнутся реакции синтеза углерода из гелия. Этот процесс идёт при более высоких
  температурах и поэтому поток энергии от ядра увеличивается, что приводит к тому, что внешние слои
  звезды начинают расширяться. Начавшийся синтез углерода знаменует новый этап в жизни звезды и
  продолжается некоторое время. Для звезды по размеру схожей с Солнцем этот процесс может занять
  около миллиарда лет.

  Дальнейшая судьба звезды зависит, главным образом, от её массы.

  \paragraph{Звезды среднего размера.} Реакции сжигания гелия очень чувствительны к температуре.
  Иногда это приводит к большой нестабильности. Возникают сильнейшие пульсации, которые в конечном
  итоге сообщают внешним слоям достаточное ускорение, чтобы быть сброшенными и превратиться в
  планетарную туманность. В центре туманности остаётся оголенное ядро звезды, в котором прекращаются
  термоядерные реакции, и оно, остывая, превращается в гелиевый белый карлик, как правило, имеющий
  массу до 0,5-0,6 солнечных и диаметр порядка диаметра Земли.

  \paragraph{Сверхмассивные звёзды.} После того, как звезда с массой большей, чем 1.4 солнечных,
  входит в стадию красного сверхгиганта, её ядро под действием сил гравитации начинает сжиматься. По
  мере сжатия увеличиваются температура и плотность, и начинается новая последовательность
  термоядерных реакций. В таких реакциях синтезируются все более тяжёлые элементы: гелий, углерод,
  кислород, кремний и железо, что временно сдерживает коллапс ядра.

  В конечном итоге, по мере образования всё более тяжёлых элементов периодической системы, из
  кремния синтезируется железо-56. На этом этапе дальнейший термоядерный синтез становится
  невозможен поскольку ядро железа-56 обладает максимальным дефектом массы и образование более
  тяжёлых ядер с выделением энергии невозможно. Поэтому когда железное ядро звезды достигает
  определённого размера, то давление в нём уже не в состоянии противостоять тяжести наружных слоёв
  звезды.  То, что происходит в дальнейшем, пока не ясно до конца, но, в любом случае, происходящие
  процессы в считанные секунды приводят к взрыву сверхновой звезды невероятной силы.

  Известно, что в некоторых сверхновых сильная гравитация в недрах сверхгиганта заставляет электроны
  поглотиться атомным ядром, где они, сливаясь с протонами, образуют нейтроны. Этот процесс
  называется нейтронизацией. Электромагнитные силы, разделяющие близлежащие ядра, исчезают. Ядро
  звезды теперь представляет собой плотный шар из атомных ядер и отдельных нейтронов.

  Если же масса звезды ещё больше, более десяти-двадцати солнечных, то коллапс будет продолжаться до
  тех пор, пока она не превратится в \emph{чёрную дыру}. Рассмотрим, что понимают под этим понятием.

  \subsection{Определение термина}

  \define{Чёрная дыра}~--- область пространства, в которой гравитационное притяжение настолько
  сильно, что ни вещество, ни излучение не могут эту область покинуть. Для находящихся там тел
  вторая космическая скорость (скорость убегания) должна была бы превышать скорость света, что
  невозможно, поскольку ни вещество, ни излучение не могут двигаться быстрее света. Поэтому из
  чёрной дыры ничто не может вылететь. Границу области, за которую не выходит свет, называют
  «горизонтом событий», или просто «горизонтом» чёрной дыры.

  \paragraph{Вторая космическая скорость.} Напомним, как получается формула второй космической
  скорости. Для получения формулы второй космической скорости удобно обратить задачу — спросить,
  какую скорость получит тело на поверхности планеты, если будет падать на неё из бесконечности.
  Очевидно, что это именно та скорость, которую надо придать телу на поверхности планеты, чтобы
  вывести его за пределы её гравитационного влияния.

  Запишем затем закон сохранения энергии: вначале, на бесконечности, и кинетическая, и потенциальная
  энергия тела равна нулю. Выбирая в качестве второго момента тот, когда оно окажется на расстоянии
  $R$ со скоростью $v_2$, получаем для закона сохранения энергии:
    \[
      \frac{mv_2^2}{2} - G\frac{mM}{R} = 0,
    \]
  откуда, решая относительно $v_2$, получаем:
    \begin{equation}\label{eq:v2}
      v_2 = \sqrt{2G\frac{M}{R}}.
    \end{equation}

  \paragraph{Гравитационный радиус в классической механике.} Применим к полученной формуле условие
  $v_2 > c$, чтобы найти, как должны быть связаны радиус и масса небесного тела, чтобы оно обладало
  свойствами чёрной дыры. Возведём в квадрат формулу \eqref{eq:v2}:
    \[
      v_2^2 = 2G\frac{M}{R},
    \]
  подставим в неё условие $c < v_2$, т.~е. $c^2 < v_2^2$:
    \[
      c^2 < 2G\frac{M}{R},
    \]
  и, наконец, выразим $R$
    \[
      \boxed{
        R < 2G\frac{M}{c^2}.
      }
    \]
  Мы получили для тела массы $M$ такой радиус $R$, при котором ничто не сможет улететь с поверхности
  такого тела, даже свет. Он называется \define{гравитационным радиусом}.

  Разобравшись с понятием чёрной дыры, рассмотрим историю его появления и развития в науке.

  \section{История}

  В истории представлений о чёрных дырах условно можно выделить три периода:
  \begin{enumerate}
    \item Начало первого периода связано с опубликованной в 1784 году работой Джона Мичелла,
      в~которой был изложен расчёт массы для недоступного наблюдению объекта.
    \item Второй период связан с развитием общей теории относительности, стационарное решение
      уравнений которой было получено Карлом Шварцшильдом в 1915 году.
    \item Публикация в 1975 году работы Стивена Хокинга, в которой он предложил идею об излучении
    чёрных дыр, начинает третий период.
  \end{enumerate}

  Далее мы рассмотрим эти периоды подробнее.

  \subsection{<<Чёрная звезда>> Мичелла}

  Английский геофизик и астроном Джон Мичелл (J.Michell, 1724--1793) предположил, что в природе могут
  существовать столь массивные звезды, что даже луч света не способен покинуть их поверхность. 

  Концепция массивного тела, гравитационное притяжение которого настолько велико, что скорость,
  необходимая для преодоления этого притяжения (вторая космическая скорость), равна или превышает
  скорость света, впервые была высказана в 1784 году Джоном Мичеллом в письме, которое он послал в
  Королевское общество. Письмо содержало расчёт, из которого следовало, что для тела с радиусом в
  500 солнечных радиусов и с плотностью Солнца вторая космическая скорость на его поверхности будет
  равна скорости света. Таким образом, свет не сможет покинуть это тело, и оно будет невидимым.
  Мичелл предположил, что в космосе может существовать множество таких недоступных наблюдению объектов.
  Поскольку понятия о предельности скорости света ещё не было, то Мичелл не рассматривал их как
  объекты которые невозможно покинуть~--- только как невидимые.
  
  В 1796 году Лаплас включил обсуждение этой идеи в свой труд «Exposition du Systeme du Monde»,
  однако в последующих изданиях этот раздел был опущен. Тем не менее, именно благодаря Лапласу эта
  мысль получила некоторую известность. Так родилась концепция «ньютоновской» черной дыры.

  \paragraph{После Мичелла.} 
  На протяжении XIX века идея тел, невидимых вследствие своей массивности, не вызывала большого
  интереса у учёных. Это было связано с тем, что в рамках классической физики скорость света не
  имеет фундаментального значения. Однако в конце XIX — начале XX века было установлено, что
  сформулированные Дж. Максвеллом законы электродинамики, с одной стороны, выполняются во всех
  инерциальных системах отсчёта, а с другой стороны, не обладают инвариантностью относительно
  преобразований Галилея. Это означало, что сложившиеся в физике представления о характере перехода
  от одной инерциальной системы отсчёта к другой нуждаются в значительной корректировке.

  В ходе дальнейшей разработки электродинамики Г. Лоренцем была предложена новая система
  преобразований пространственно-временных координат (известных сегодня как преобразования Лоренца),
  относительно которых уравнения Максвелла оставались инвариантными. Развивая идеи Лоренца, А.
  Пуанкаре предположил, что все прочие физические законы также инвариантны относительно этих
  преобразований.

  В 1905 году А. Эйнштейн использовал концепции Лоренца и Пуанкаре в своей специальной теории
  относительности (СТО), в которой роль закона преобразования инерциальных систем отсчёта
  окончательно перешла от преобразований Галилея к преобразованиям Лоренца. Классическая
  (галилеевски-инвариантная) механика была при этом заменена на новую, лоренц-инвариантную
  релятивистскую механику. В рамках последней скорость света оказалась предельной скоростью, которую
  может развить физическое тело, что \emph{радикально изменило значение чёрных дыр в теоретической физике}.

  Однако ньютоновская теория тяготения (на которой базировалась первоначальная теория чёрных дыр) не
  является лоренц-инвариантной. Поэтому она не может быть применена к телам, движущимся с
  околосветовыми и световыми скоростями. Лишённая этого недостатка релятивистская теория тяготения
  была создана, в основном, Эйнштейном (сформулировавшим её окончательно к концу 1915 года) и
  получила название общей теории относительности (ОТО). Именно на ней и основывается современная
  теория астрофизических чёрных дыр.

  \subsection{Чёрные дыры в ОТО}

  По своему характеру ОТО является геометрической теорией. Она предполагает, что гравитационное поле
  представляет собой проявление искривления пространства-времени (которое, таким образом,
  оказывается псевдоримановым, а не псевдоевклидовым, как в специальной теории относительности).
  Связь искривления пространства-времени с характером распределения и движения заключающихся в нём
  масс даётся основными уравнениями теории — уравнениями Эйнштейна.
  \[
    R_{ab} - \frac{R}{2}  g_{ab} + \Lambda g_{ab} = \frac{8 \pi G}{c^4} T_{ab},
  \]
  где $R_{ab}$~--- тензор Риччи, получающийся из тензора кривизны пространства-времени $R_{abcd}$
  посредством свёртки его по паре индексов, $R$ — скалярная кривизна, то есть свёрнутый тензор
  Риччи, $g_{ab}$ — метрический тензор, $\Lambda$ — космологическая постоянная, а $T_{ab}$
  представляет собой тензор энергии-импульса материи.

  Решением Шварцшильда точно описывается изолированная невращающаяся, незаряженная и не испаряющаяся
  чёрная дыра (это сферически симметричное решение уравнений
  Эйнштейна в вакууме). Её горизонт событий — это сфера, радиус которой, определённый из её площади
  по формуле $S=4\pi r^2,$ называется гравитационным радиусом или радиусом Шварцшильда.

  Все характеристики решения Шварцшильда однозначно определяются одним параметром — массой. Так,
  гравитационный радиус чёрной дыры массы $M$ равен
  \[
      \boxed{
        r_s = 2G\frac{M}{c^2}.
      }
    \]
  где G — гравитационная постоянная, а c — скорость света. Чёрная дыра с массой, равной массе Земли,
  обладала бы радиусом Шварцшильда около 9 мм (то есть Земля могла бы стать чёрной дырой, если бы
  кто-либо смог сжать её до такого размера). Для Солнца радиус Шварцшильда составляет примерно 3 км.

  Объекты, размер которых наиболее близок к своему радиусу Шварцшильда, но которые ещё не являются
  чёрными дырами, — это нейтронные звёзды

  \subsection{Излучение Хокинга}

  Излучением Хокинга называют гипотетический процесс испускания разнообразных элементарных частиц,
  преимущественно фотонов, чёрной дырой. Температуры известных астрономам чёрных дыр слишком малы,
  чтобы излучение Хокинга от них можно было бы зафиксировать — массы дыр слишком велики. Поэтому до
  сих пор эффект не подтверждён наблюдениями.

  Согласно ОТО, при образовании Вселенной могли бы рождаться первичные чёрные дыры, некоторые из
  которых (с начальной массой $10^{12}$ кг) должны были бы заканчивать испаряться в наше время. Так как
  интенсивность испарения растёт с уменьшением размера чёрной дыры, то последние стадии должны быть
  по сути взрывом чёрной дыры. Пока таких взрывов зарегистрировано не было.

  Известно о попытке исследования «излучения Хокинга» на основе модели — аналога горизонта событий
  для белой дыры, в ходе физического эксперимента, проведенного исследователями из Миланского
  университета

\end{document}
