% Отключаем subsection!
\PassOptionsToPackage{subsection=false}{beamerouterthememiniframes}
\documentclass[usenames,dvipsnames,pdftex,unicode,hidelinks]{beamer}
  \usepackage{cmap}
  \usepackage[T2A]{fontenc}
  \usepackage[utf8]{inputenc}
  \usepackage[english,russian]{babel}
  \usepackage{wasysym}
  \usepackage{mathtext} % для кириллицы в формулах
    \DeclareSymbolFont{T2Aletters}{T2A}{cmr}{m}{it} % кириллица в формулах курсивом

  \usepackage{tikz}
    \usetikzlibrary{positioning,fit,backgrounds}
  \graphicspath{{../img/}{../../img/}}

  \usepackage{numprint}
    % алиас и настройки для numprint
    \let\num\numprint
    \npthousandsep{\,}
    \npthousandthpartsep{}
    \npdecimalsign{,}

  % add frame number to footline
  \let\oldmacro\insertshorttitle
  \renewcommand*\insertshorttitle{
    \oldmacro\hfill
    -\,\insertframenumber\,- % TODO: temporary comment total count % \,/\,\inserttotalframenumber
  }
  % TODO HACK: вместо института выводим автора
  \renewcommand*\insertshortinstitute{
    \insertshortauthor
  }

  % hide navigation symbols
  \beamertemplatenavigationsymbolsempty

  \usetheme{Szeged}%{Warsaw}
  \usecolortheme{seahorse}
  \usefonttheme{structurebold}
  \useinnertheme{rounded}
  % Патчим некоторые цвета
  %\setbeamertemplate{background canvas}[vertical shading][bottom=black!60!blue,top=black!80!blue]
  \setbeamercolor{block title}{bg=title.bg}
  \setbeamercolor{block body}{bg=title.bg!60}
  % Добавляем "action environment" для смены цвета bullet'а одного item'а (http://tex.stackexchange.com/a/14366/24732)
  \newenvironment{newenv}{\only{\setbeamercolor{item}{fg=red}}}{}
  \newenvironment{doneenv}{\only{\setbeamercolor{item}{fg=OliveGreen}}}{}
  \newcommand{\done}{\textcolor{OliveGreen}{\textbf{\checked}}}

  \setbeamercovered{transparent}

  \newcommand{\muted}[1]{\textcolor{gray}{#1}}
  \let\softalert\textbf  % более мягкий alert, чем выделение красным цветом

  \let\vect\vec    % единое выделение векторов (стрелкой)
  \let\matx\mathbf % единое выделение матриц (полужирным)
  \let\transposed\top % единый знак транспонирования (U+22A4 down tack)
  \newcommand{\conj}[1]{#1^*} % единое обозначение комплексного сопряжения (черта сверху)
  \renewcommand{\le}{\leqslant} % <= с наклонной нижней перекладиной
  \renewcommand{\ge}{\geqslant} % >= с наклонной нижней перекладиной
  \renewcommand{\phi}{\varphi} % phi завитушкой

  \newcommand{\todo}[1]{\textbf{\textcolor{red}{TODO: #1}}}
  \newcommand{\GPU}{\alert{GPU}} % короткая запись для выделенного GPU (встречается несколько раз)
  \newcommand{\twocol}[1]{\multicolumn{2}{c|}{#1}}

  % цвета текста для выделения
  \newcommand{\grn}[1]{\textcolor{ForestGreen}{#1}}
  \newcommand{\ppl}[1]{\textcolor{RoyalPurple}{#1}}
  \newcommand{\blu}[1]{\textcolor{NavyBlue}{#1}}

\title[Система моделирования пластических деформаций биологических объектов]{Информационная система моделирования динамики пластических деформаций биологических объектов}
\author[Иван Новиков]{Новиков Иван Александрович}

\institute{Н1: Информационные технологии. \textbf{2\textsuperscript{й} год}}

\date{ \the\year{} г. }

\begin{document}

  % Local background must be enclosed by curly braces for grouping.
  {
    \usebackgroundtemplate{
      \begin{tikzpicture}[show background rectangle, inner frame sep=-2mm, background rectangle/.style={ draw=none }]
        \node (t) at (0,0) {};% HACK :(
       \node[] at (2, -6.5) {
         \includegraphics[height=4.5cm]{screen-title-left}
       };
       \node[] at (10.6, -6.5) {% подгон :(
         \includegraphics[height=4.5cm]{screen-title-right}
       };
      \end{tikzpicture}
    }
    \begin{frame}[plain]
      \titlepage
    \end{frame}
  }

  \section{Постановка задачи}
  \subsection{Постановка задачи} % NB: subsection отключены! Эти имена не будут отображаться
  \begin{frame}{Применения моделирования деформаций}
    % Уменьшаем margin сверху (см. http://tex.stackexchange.com/a/73522/24732)
    \vspace{-2\baselineskip}
    \begin{columns}[c]
      \begin{column}{0.5\textwidth}<-1>
        \begin{center}
          Компьютерные игры
          \only<1>{ \includegraphics[height=0.3\textheight]{game} }
          % TODO честная полупрозрачность
          \only<2>{ \includegraphics[height=0.3\textheight]{game-phantom} }

          Спецэффекты в кино

          \only<1>{ \includegraphics[height=0.3\textheight]{movie} }
          % TODO честная полупрозрачность
          \only<2>{ \includegraphics[height=0.3\textheight]{movie-phantom} }
        \end{center}
      \end{column}
      \begin{column}{0.5\textwidth}<-2>
        \begin{center}
          Обучающие тренажёры

          \includegraphics[height=0.3\textheight]{trainer}

          Системы авт. проектирования

          \includegraphics[height=0.3\textheight]{cad}
        \end{center}
      \end{column}
    \end{columns}
  \end{frame}
  \begin{frame}{Моделирование в реальном времени}
    В приложениях, связанных с визуализацией:
    \begin{enumerate}
      \item \alert{Мягкое} реальное время.
      \item Комфорт пользователя: FPS $\ge 30$.
      \item За $\Delta t \le 33$~мс --- \alert{все} расчёты очередного кадра.
      \item На моделирование деформаций отводится \alert{$\Delta t' \approx 3..10$~мс}.
    \end{enumerate}

    \vspace{5mm}

    \begin{center}
      \begin{tikzpicture}[scale=1]
        \fill[Gray!50] (0,0) circle (1);
        \node at (0,0) {1 сек};

        \fill[Gray!50] (3, 0) circle (1);
        \fill[OliveGreen] (3, 0) -- +(0, 1) arc (90:78:1) -- cycle;
        \foreach \a in {6,18,...,359}
          \draw[thin, Gray!75] (3, 0) -- +(\a:1);
        \node[below,OliveGreen] at (3,0) {$\frac1{30}$ с};

        \fill[OliveGreen] (6, 0) circle (1);
        \fill[Red!50] (6, 0) -- +(0, 1) arc (90:54:1) -- cycle;
        \node[below,Red!50] at (6,0) {3 мс};
      \end{tikzpicture}
    \end{center}
  \end{frame}

  \begin{frame}{Актуальность}
    Разработка такой системы актуальна:
    \begin{itemize}
      \item применение систем виртуальной реальности \alert{в~обучении};
        % NB: сказать про перспективы 3D-печати тканей,органов...
      \item потребность в \alert{САПР} протезов, имплантатов;
      \item отсутствие \alert{отечественных} аналогов.
    \end{itemize}

    \vspace{0.5cm}

    \uncover<2->{
    Недостатки зарубежных (ANSYS, DEFORM-3D, COMSOL):
    \begin{itemize}
      \item не адаптированы для \alert{биологических} объектов;
      \item нельзя применить к расчётам в \alert{реальном} времени (на PC);
      \item высокая цена.
    \end{itemize}
    }
  \end{frame}

  \begin{frame}{Цель}
    \begin{block}{Цель проекта}
      Разработка \alert{системы моделирования} пластических деформаций биологических объектов
      \alert{в~реальном времени}.
      % (которая может быть использована как отдельно, так и в качестве
      % \alert{модуля} для встраивания в стороннее приложение)
    \end{block}

    \vspace{0.5cm}

    Задачи:
    \begin{enumerate}
      \item Разработка \grn{алгоритма} моделирования
      \item Реализация \grn{его} в виде \ppl{программной библиотеки}
      \item Разработка на \ppl{её} основе \blu{интерактивного приложения}
      \item<new@1-> Интеграция \ppl{и}\blu{х} с 3D-редактором
    \end{enumerate}
  \end{frame}

  \section{Содержание проекта}
  \subsection{Содержание проекта}
  \begin{frame}{Новизна}
    \begin{enumerate}
      \setlength{\itemsep}{4mm}
      \item Производительный алгоритм моделирования деформаций:

        \includegraphics[height=15mm]{meshless}\\[-2mm]
        {
          \tiny \textcopyright{} \href{http://dx.doi.org/10.1145/1186822.1073216}{M.\,M\"{u}ller et al. Meshless deformations based on shape matching}
        }
        \begin{itemize}
          \item объединение физических методов с \alert{геометрическими};
          \item \alert{минимум ограничений} на 3D-модель объекта
          \item применение графического процессора (\GPU).
        \end{itemize}
      \item \alert{Гибкий} программный интерфейс (API).
      \item Простота \alert{конфигурации}.
    \end{enumerate}
  \end{frame}

  \begin{frame}{Назначение}
    Две формы продукта:

    \begin{enumerate}
      \setlength{\itemsep}{5mm}
      \item Модуль для встраивания в стороннее приложение
        \begin{itemize}
          \item Используется разработчиком (в более крупной системе)
          \item Предоставляет ей функции моделирования деформаций
        \end{itemize}
        \softalert{Основные потребители}: разработчики САПР
      \item Автономное приложение \textcolor{Gray}{(основанное на этой библиотеке)}
        \begin{itemize}
          \item Используется конечным пользователем
          \item Может применяться для обучения
        \end{itemize}
        \softalert{Основные потребители}: высшие учебные заведения
    \end{enumerate}
  \end{frame}

  \begin{frame}{Потенциальные заказчики}
    \begin{itemize}
      \setlength{\itemsep}{5mm}
      \item Учебные заведения

        \foreach \name in {ksma, kubsu, kubstu} {
          \hspace{5mm}\includegraphics[height=15mm]{\name}
        }
      \item Научные организации

        \foreach \name in {ramn, sbras} {
          \hspace{5mm}\includegraphics[height=10mm]{\name}
        }
      \item Разработчики САПР

        \foreach \name in {nanocad, ascon, rusapr} {
          \hspace{5mm}\includegraphics[height=10mm]{\name}
        }
    \end{itemize}
  \end{frame}

  \section{План проекта}
  \subsection{План проекта}
  \begin{frame}{Календарный план}
    1\textsuperscript{й} год:
    \begin{enumerate}
      \item<done@1-> Разработка ядра системы \done
        \begin{itemize}
          \item Алгоритм моделирования деформаций. \done
          \item Программный интерфейс (API) \done
        \end{itemize}
      \item<done@1-> Разработка графической оболочки \done
        \begin{itemize}
          \item Графический 3D-движок \done
          \item Пользовательский интерфейс (GUI) \done
        \end{itemize}
    \end{enumerate}
    \uncover<2->{
      2\textsuperscript{й} год:
      \begin{itemize}
        \item<new@1-> Совершенствование алгоритма моделирования
        \item<new@1-> Более полное использование параллельных вычислений
        \item<new@1-> Расширение функций GUI
        \item<new@1-> Интеграция с 3D-редакторами
      \end{itemize}
    }
  \end{frame}

  \begin{frame}{Результаты 1 года}
    \begin{center}
      \only<1>{ Графический интерфейс \\ \includegraphics[height=0.7\textheight]{screen-gui} }
      \only<2>{ 1. До удара \\ \includegraphics[height=0.7\textheight]{screen-0} }
      \only<3>{ 2. Вскоре после удара \\ \includegraphics[height=0.7\textheight]{screen-1} }
      \only<4>{ 3. После удара \\ \includegraphics[height=0.7\textheight]{screen-2} }
    \end{center}
  \end{frame}

  \begin{frame}{Конечный результат}
    \begin{itemize}
      \setlength{\itemsep}{5mm}
      \item Алгоритм моделирования пластических деформаций биологических объектов в
        реальном времени
      \item Программный модуль (библиотека), реализующий алгоритм, с гибким интерфейсом (API)
      \item Графическое приложение для запуска моделирования \\(на основе этой библиотеки)
      \item Вспомогательные инструменты
    \end{itemize}
  \end{frame}


  % Заканчиваем последний section, чтобы заключение, "спасибо" и запасные слайды к нему не
  % относились и не отображались в навигации сверху
  \section{}

  \begin{frame}[plain]
    \begin{center}
      { \Huge Спасибо за внимание! }

      \vspace{1cm}

      Иван Новиков\\
      \url{http://about.me/moonlighter}\\
      \href{mailto:nia.afti@gmail.com}{\nolinkurl{nia.afti@gmail.com} }

    \end{center}
  \end{frame}

  % Запасные слайды

  \begin{frame}{Производительность}
    \begin{center}
      \begin{columns}[c]
        \begin{column}{0.27\textwidth}
          \includegraphics[width=\textwidth]{heart}
        \end{column}
        \begin{column}{0.7\textwidth}
          \begin{center}
            \underline{Тестовый объект:}

            \grn{Физическая сетка : \textbf{\num{9202} точки}}

            \blu{Графическая 3D-модель: \textbf{\num{92026} вершин}}
            \vspace{5mm}

            {
              \footnotesize
              Время \grn{моделирования}/\blu{обновления}, мс:
              \begin{tabular}{|>{\bfseries}r|c|c|c|c|}
                \hline
                Алгоритм             & \twocol{Линейный}  & \twocol{Квадратичный} \\\hline
                Потоков              & 1 & 4 & 1 & 4 \\\hline
                \grn{Физика} (CPU)   & \grn{2,96}  & \grn{2,04} & \grn{4,33}  & \grn{2,57}  \\\hline
                \blu{Графика} (CPU)  & \blu{10,98} & \blu{3,82} & \blu{14,75} & \blu{4,83}  \\\hline
                \blu{Графика} (\GPU) & \twocol{\blu{0,22}}      & \twocol{\blu{0,24}}\\\hline
              \end{tabular}
            }
          \end{center}
        \end{column}
      \end{columns}
      \vspace{5mm}

      \softalert{Итого:} \grn{физика: от \softalert{\num{3500}} тчк/мс},
                         \blu{графика: до \softalert{\num{380000}} тчк/мс}
      \footnote{
        Intel Core i5-2500K 3.3 GHz, NVIDIA GeForce GTX 560 Ti 448
      }
    \end{center}
  \end{frame}

  \begin{frame}{Использование в обучении}
    \begin{tikzpicture}[scale=1.5]
      \tikzstyle{block} = [rectangle, draw=blue, rounded corners, fill=cyan!10,
                           text centered, text width=2.3cm, font=\footnotesize, minimum height=4em]
      \tikzstyle{bblck} = [block, minimum height=3em, draw=Green, fill=LimeGreen!30]
      \tikzstyle{tblck} = [block, minimum height=3em, draw=Plum, fill=BlueViolet!10]
      \tikzstyle{mblck} = [block, minimum height=3em, draw=RedOrange, fill=Red!10]
      \tikzstyle{last} = [text width=3cm, font=\scriptsize]

      \node[block] (root) {Высшее образование};
      \node[tblck] (tech) [right=1.0cm of root] {Биотехнологии};
      \node[bblck] (bio)  [above=0.3cm of tech] {Биология};
      \node[mblck] (med)  [below=0.3cm of tech] {Медицина};

      \node[bblck,last] (bio1) [right=1.0cm of bio]  {Моделирование тканей};
      \node[bblck,last] (bio2) [above=0.2cm of bio1] {Механика движения};

      \node[tblck,last] (tech1)[right=1.0cm of tech] {Имплантация, протезирование};

      \node[mblck,last] (med1) [right=1.0cm of med]  {Моделирование органов в движении};
      \node[mblck,last] (med2) [below=0.2cm of med1] {Моделирование воздействий на ткани};

      \foreach \nd/\n in {bio/2,tech/1,med/2} {
        \draw[->, thick] (root) -- (\nd);
        \foreach \x in {1,...,\n}
          \draw[->, thick] (\nd) -- (\nd\x);
      }
    \end{tikzpicture}
  \end{frame}

  \begin{frame}{Другие области применения}
    \begin{itemize}
      \setlength{\itemsep}{5mm}
      \item Компьютерное моделирование крэш-тестов
        \begin{itemize}
          \item Существующие системы моделируют только деформацию автомобиля
        \end{itemize}

      \item Компьютерные игры
        \begin{itemize}
          \item Отечественные производители игр пока сильно отстают от зарубежных
        \end{itemize}

      \item Другие системы виртуальной реальности
        \begin{itemize}
          \item Кроме тренажёров также развиваются виртуальные музеи, интерактивные образовательные
            программы, ...
        \end{itemize}
    \end{itemize}
  \end{frame}

\end{document}

