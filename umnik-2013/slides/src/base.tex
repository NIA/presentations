% Отключаем subsection!
\PassOptionsToPackage{subsection=false}{beamerouterthememiniframes}
\documentclass[usenames,dvipsnames,pdftex,unicode,hidelinks]{beamer}
  \usepackage{cmap}
  \usepackage[T2A]{fontenc}
  \usepackage[utf8]{inputenc}
  \usepackage[english,russian]{babel}
  \usepackage{wasysym}
  \usepackage{mathtext} % для кириллицы в формулах
    \DeclareSymbolFont{T2Aletters}{T2A}{cmr}{m}{it} % кириллица в формулах курсивом

  \usepackage{tikz}
    \usetikzlibrary{positioning,fit,backgrounds}
  \graphicspath{{../img/}{../../img/}}

  % add frame number to footline
  \let\oldmacro\insertshorttitle
  \renewcommand*\insertshorttitle{
    \oldmacro\hfill
    -\,\insertframenumber\,- % TODO: temporary comment total count % \,/\,\inserttotalframenumber
  }

  % hide navigation symbols
  \beamertemplatenavigationsymbolsempty

  \usetheme{Szeged}%{Warsaw}
  \usecolortheme{seahorse}
  \usefonttheme{structurebold}
  \useinnertheme{rounded}
  % Патчим некоторые цвета
  %\setbeamertemplate{background canvas}[vertical shading][bottom=black!60!blue,top=black!80!blue]
  \setbeamercolor{block title}{bg=title.bg}
  \setbeamercolor{block body}{bg=title.bg!60}

  \setbeamercovered{transparent}

  \newcommand{\muted}[1]{\textcolor{gray}{#1}}

  \newcommand{\vect}[1]{\vec{#1}} % единое выделение векторов (стрелкой)
  \newcommand{\matx}[1]{\mathbf{#1}} % единое выделение матриц (полужирным)
  \newcommand{\transposed}{\top} % единый знак транспонирования (U+22A4 down tack)
  \newcommand{\conj}[1]{#1^*} % единое обозначение комплексного сопряжения (черта сверху)
  \renewcommand{\le}{\leqslant} % <= с наклонной нижней перекладиной
  \renewcommand{\ge}{\geqslant} % >= с наклонной нижней перекладиной
  \renewcommand{\phi}{\varphi} % phi завитушкой

  \newcommand{\todo}[1]{\textbf{\textcolor{red}{TODO: #1}}}

\title[Система моделирования деформаций биологических объектов]{Информационная система моделирования динамики пластических деформаций биологических объектов}
\author[Иван Новиков]{Новиков Иван Александрович}

\institute{Кубанский Государственный Университет}

\date{ 23 октября 2013 г. }

\begin{document}
  
  % Local background must be enclosed by curly braces for grouping.
  {
    \usebackgroundtemplate{
      \newcounter{cntShader}
      \begin{tikzpicture}[show background rectangle, inner frame sep=2mm, background rectangle/.style={ draw=none }]
        \foreach \x in {-4, ..., 21} {
          \foreach \y in {-3, ..., 15} {
            \pgfmathsetcounter{cntShader}{ 500 / (abs(\x) + 1) / (abs(\y) + 1)}
            \shade[shading=radial,inner color = blue!\thecntShader] (\x*0.5, \y*0.5) circle (0.1);
          }
        }
      \end{tikzpicture}
    }
    \begin{frame}[plain]
      \titlepage
    \end{frame}
  }

  \section{Постановка задачи}
  \subsection{Постановка задачи} % NB: subsection отключены! Эти имена не будут отображаться
  \begin{frame}{Моделирование деформаций}
    \todo{(здесь будет о том, что такое моделирование деформаций и зачем. Картинки)}
  \end{frame}
  \begin{frame}{Моделирование в реальном времени}
    \todo{(о том, что мы в данном случае понимаем под реальным временем, и что это даёт)}
  \end{frame}

  \begin{frame}{Актуальность}
    \todo{Это актуально так как:}
    \begin{itemize}
      \item \todo{причина}
      \item \todo{причина}
      \item Отечественных аналогов нет
    \end{itemize}
    
    Недостатки зарубежных аналогов (ANSYS, DEFORM-3D)
    \begin{itemize}
      \item недостаточная скорость для реального времени
      \item актуальность
      \item актуальность
    \end{itemize}
  \end{frame}

  \begin{frame}{Цель}
    \begin{block}{Цель проекта}
      Разработка программной системы, осуществляющей моделирование пластических деформаций
      биологических объектов в реальном времени, которая может быть использована как отдельно, так и
      в качестве модуля для встраивания в стороннее приложение.
    \end{block}
    Задачи:
    \begin{enumerate}
      \item \todo{задача}
      \item \todo{задача}
      \item \todo{задача}
    \end{enumerate}
  \end{frame}

  \section{Содержание проекта}
  \subsection{Содержание проекта}
  \begin{frame}{Научная основа}

  \end{frame}

  \begin{frame}{Имеющийся задел}

  \end{frame}

  \begin{frame}{Потенциальные заказчики}

  \end{frame}

  \begin{frame}{Коммерциализация}

  \end{frame}

  \section{План проекта}
  \subsection{План проекта}
  \begin{frame}{План}

  \end{frame}

  \begin{frame}{Перспективы и риски}

  \end{frame}

  \begin{frame}{Ожидаемый результат}

  \end{frame}


  % Заканчиваем последний section, чтобы заключение, "спасибо" и запасные слайды к нему не
  % относились и не отображались в навигации сверху
  \section{}

  \begin{frame}[plain]
    \begin{center}
      { \Huge Спасибо за внимание! }

      \vspace{1cm}

      Иван Новиков\\
      \url{http://about.me/moonlighter}\\
      \href{mailto:nia.afti@gmail.com}{\nolinkurl{nia.afti@gmail.com} }
      
    \end{center}
  \end{frame}

\end{document}

