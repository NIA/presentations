\documentclass[a4paper, 12pt]{extarticle}
  \usepackage{cmap}
  \usepackage[hidelinks,pdftex,unicode]{hyperref}
  \usepackage{mathtext} % для кириллицы в формулах
  \usepackage[T2A]{fontenc}
  \usepackage[utf8]{inputenc}
  \usepackage[russian]{babel}
  \usepackage{indentfirst}
  \usepackage{cite}
  \usepackage{amsmath} % для \eqref
  \usepackage{amssymb} % для \leqslant
  \usepackage{amsthm} % для \pushQED
  \usepackage[usenames,dvipsnames]{xcolor}
    \definecolor{dkblue}{rgb}{0,0,0.6}
  \usepackage[pdftex]{graphicx}
  \usepackage{subfig}
  \usepackage{tikz}
    \usetikzlibrary{positioning,fit,shapes,calc}
  \usepackage{numprint}
  \usepackage[left=1.5cm,right=1.5cm,top=1cm,bottom=1.5cm,bindingoffset=0cm]{geometry}
  \usepackage{datetime}
  \usepackage[normalem]{ulem} % красивые подчёркивания, но обычный курсив в \em, \emph
    % Хак для использования \sout итп в заголовках (http://tex.stackexchange.com/a/22415)
    \DeclareRobustCommand{\hsout}[1]{\texorpdfstring{\sout{#1}}{#1}}
  \graphicspath{{../img/}{../../img/}}
  \frenchspacing

  \DeclareSymbolFont{T2Aletters}{T2A}{cmr}{m}{it} % кириллица в формулах курсивом

  \addto\captionsrussian{
    \renewcommand\contentsname{Содержание}
    % перекрываю \refname, чтобы список литературы сам добавлял себя в оглавление
    \let\oldrefname\refname
    \renewcommand\refname{\addcontentsline{toc}{section}{\oldrefname}\oldrefname}
  }

  \newcommand{\underscore}[1]{\hbox to#1{\hrulefill}}
  \newcommand{\todo}[1]{\textbf{\textcolor{red}{TODO: #1}}}
  \newcommand{\note}[1]{\textit{Примечание: #1}}
  \newcommand{\eng}[1]{{\English #1}}

  % обёртка с моими настройками поверх figure:
  % \begin{myfigure}{подпись}{fig:label} ... \end{myfigure}
  \newenvironment{myfigure}[2]%
    {\pushQED{\caption{#1} \label{#2}} % push caption & label
     \begin{figure}[!htb]\centering } %
    {  \popQED % pop caption & label
     \end{figure}}

  % вставка картинки: \figure[params]{подпись}{file}
  % создаёт label вида fig:file
  \newcommand{\includefigure}[3][]{
    \begin{myfigure}{#2}{fig:#3}
      \includegraphics[#1]{#3}
    \end{myfigure}
  }

  % вставка subfigure внутри myfigure:
  % \subfigure[params]{подпись}{file}
  \newcommand{\subfigure}[3][]{
    \subfloat[#2]{\label{fig:#3}\includegraphics[#1]{#3}}
  }

  \newcommand{\vect}[1]{\vec{#1}} % единое выделение векторов (стрелкой)
  \newcommand{\matx}[1]{\mathbf{#1}} % единое выделение матриц (полужирным)
  \newcommand{\transposed}{\top} % единый знак транспонирования (U+22A4 down tack)
  \newcommand{\conj}[1]{#1^*} % единое обозначение комплексного сопряжения (черта сверху)
  \renewcommand{\le}{\leqslant} % <= с наклонной нижней перекладиной
  \renewcommand{\ge}{\geqslant} % >= с наклонной нижней перекладиной
  \renewcommand{\phi}{\varphi} % phi завитушкой
  \newcommand{\I}{\mathrm{I}} % Интенсивность

  \linespread{1.2}

  % русские буквы для списков и частей рисунка
  \renewcommand{\theenumii}{(\asbuk{enumii})}
  \renewcommand{\labelenumii}{\asbuk{enumii})}
  \renewcommand{\thesubfigure}{\asbuk{subfigure}}

  \setcounter{tocdepth}{3} % глубина оглавления

  \bibliographystyle{gost780u}

  \hyphenation{англ} % убрать перенос в этом сокращении

  % алиас и настройки для numprint
  \newcommand{\num}[1]{\numprint{#1}}
  \npthousandsep{\,}
  \npthousandthpartsep{}
  \npdecimalsign{,}

  \newcommand{\checkdate}[3]{({\Russian дата обращения: \formatdate{#1}{#2}{#3}})}

  % выделение ещё более важных моментов, чем \emph
  \newcommand{\strong}[1]{\textbf{#1}}
  % подчёркивание определяемого в определении
  \newcommand{\define}[1]{\uwave{#1}}

  % участок с увеличенными полями
  \newenvironment{indented}%
    { \begingroup %
        \noindent %
        \leftskip2em %
        \rightskip\leftskip }%
    { \par\endgroup }

  % выделение примеров
  \newenvironment{example}%
    { \begin{indented} %
        \color{dkgreen} %
        \small %
        \textbf{\underline{Пример:}} }%
    { \end{indented} }

  % выделение дополнительной, "продвинутой" информации, необязательной при первом прочтении
  \newenvironment{extrainfo}%
    { \begin{indented} %
        \color{dkblue} %
        \small %
        \textbf{\textcircled{\footnotesize i}} }%
    { \end{indented} }

  % пол-страничная по ширине область (для листингов)
  \newenvironment{halfpage}%
    {\noindent\begin{minipage}[h]{0.49\linewidth}} %
    {\end{minipage}\hfill}

  \newcounter{slidescount}
  % Вставляет номер слайда (употреблять в начале абзаца)
  \newcommand{\slide}{
    \paragraph{\large \textcircled{\footnotesize \arabic{slidescount}} }%
    \stepcounter{slidescount}
  }
  % Вставляет знак перехода к следующему шагу в пределах слайда
  \newcommand{\click}{\colorbox{yellow}{$\Longrightarrow$} }

  \author{Иван\,Новиков}
  \title{Информационная система моделирования динамики пластических деформаций биологических объектов}
  \date{23 октября 2013~г.}

\begin{document}

  % \maketitle

  \slide Развитие компьютеров и рост их производительности открывает всё более широкие
  возможности для применения компьютерного моделирования. Моделирование физических явлений позволяет
  испытывать различные изделия прямо в процессе их проектирования, а~также является важной
  составляющей при создании виртуальной реальности.

  \slide Рассмотрим области применения конкретно моделирования деформаций. Его используют приложения
  виртуальной реальности (компьютерные игры и обучающие тренажёры) для увеличения реалистичности,
  с~той же целью системы моделирования могут применяться при создании спец-эффектов для фильмов.
  Более точные расчёты деформаций производят системы автоматизированного проектирования для
  испытания созданных деталей. \click В настоящее время более актуальными являются направления,
  указанные справа, о чём будет сказано далее. При этом, характерной особенностью приложений
  обучающих тренажёров, как и компьютерных игр, является необходимость работы в~реальном времени.
  Остановимся чуть подробнее на этом требовании.

  \slide Термин <<реальное время>> применяется широко, и его смысл зависит от области применения.
  В~случае приложений виртуальной реальности речь не идёт о жёстком реальном времени (как, например,
  в системе управления электростанцией), но о мягком, т.~е. превышение задержки означает лишь снижение
  качества работы. При этом конкретные временные ограничения исходят из особенностей глаза человека:
  анимация на экране будет казаться плавной, только если частота кадров превышает 30 кадров/с. Таким
  образом, все расчёты очередного кадра должны занимать не более $\frac{1}{30}$ секунды.
  Но~поскольку расчёт деформаций~--- не единственная задача, выполняемая на каждом кадре, то на~него
  отводится лишь часть этого интервала, как правило, от 3 до 10 мс.

  \slide Перейдём к актуальности работ по моделированию деформаций биологических объектов.
  Уже говорилось об обучающих тренажёрах, они широко применяются для обучения водителей, лётчиков,
  военных, но особенно полезны могут стать развивающиеся сейчас тренажёры для студентов-медиков,
  позволяющие обучать их без риска для людей. Кроме того, развитие технологий трёхмерной печати уже
  сейчас позволяет создать объект любой формы по его трёхмерной модели. Возможность испытать
  физические свойства объекта ещё до изготовления очень перспективно для производства различных
  протезов и биоимплантатов. \click Особенно актуальным это становится по причине отсутствия
  российских аналогов у предлагаемой системы. Системы моделирования пластических деформаций
  существуют на зарубежном рынке, но их главными недостатками является то, что они адаптированы для
  моделирования металлических деталей, а не биологических объектов, а также то, что их невозможно
  использовать в приложениях виртуальной реальности.

  \slide Целью данного проекта является разработка такой системы, которая не имела бы этих
  недостатков и могла бы быть применена в указанных отраслях, как в качестве отдельного приложения,
  так и в качестве модуля для создания новых приложений на её основе. Для этого ставятся задачи
  разработки алгоритма, реализующей его программной библиотеки, и использующей её интерактивного
  3D-приложения.

  \slide Особенностями данной системы станут применение собственных наработок по моделированию
  деформаций, включая использование быстрых геометрических методов, их адаптация для моделирования
  биологических объектов, а также наличие гибкого интерфейса, облегчающего встраивание системы
  в~стороннее приложение.

  \slide На данный момент уже имеется задел для реализации этого проекта в виде нескольких
  публикаций, авторских свидетельств о регистрации программ и работающего прототипа будущей системы.
  Поговорим о нём подробнее.

  \slide Прототип представляет из себя интерактивное приложение, отображающие 3D-модель средствами
  DirectX и позволяющее наносить удары по модели с разных сторон. \click После активации удара
  в~реальном времени отображается изменение формы, \click в итоге мы видим деформированный объект,
  причём кроме изменений формы можно наблюдать визуализацию различных событий: например, здесь
  красным окрасились области, деформация которых превысила заданный порог.

  \slide После того, как система будет полностью реализована, её потребителями могут стать следующие
  организации: высшие учебные заведения, обучающие студентов медиков и биологов, исследовательские
  институты, а также компании-разработчики ПО, а именно систем автоматизированного проектирования.

  \slide Таким образом, конечный продукт будет доступен в двух формах: в виде модуля для встраивания в
  стороннее приложение, которое разработчики САПР смогут приобретать для разработке на его основе
  своих продуктов, и в виде готового приложения, который будет применяться в качестве обучающего
  тренажёра.

  \slide Календарный план проекта напрямую вытекает из задач, озвученных ранее: разработки
  алгоритма, реализующей его библиотеки и готового приложения. Отдельный этап отводится на
  тестирование стабильности и точности моделирования. \click При этом одним годом развитие этого
  проекта не ограничивается: в дальнейшем можно будет заниматься дальнейшей оптимизацией алгоритмов,
  разработкой дополнительных инструментов и добавлением возможностей расширения в систему.

  \slide Дальнейшие перспективы проекта связаны с сотрудничеством с основными потребителями: с
  пользователями специализированных обучающих тренажёров~--- по более глубокой интеграции системы в
  эти тренажёры, с разработчиками САПР~--- по созданию совместных продуктов. Риски, которые могут
  быть связаны с аппаратной несовместимостью, консервативностью потребителей и нарушением
  интеллектуальной собственности будут учитываться соответствующим образом.

  \slide Таким образом, конечный результат будет представлен разработанным алгоритмом моделирования
  деформаций, программным модулем, позволяющим лёгкое встраивание его в существующее приложение,
  а~также автономным приложением на его основе. В дальнейшем возможно также добавление в систему
  вспомогательных инструментов, расширяющих возможности пользователя по заданию входных данных для
  моделирования.

\end{document}
